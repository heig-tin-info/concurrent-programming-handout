
\chapter{Temps d'exécution}\label{sec:temps}

\startchapter
%\section{}


\lettrine[lines=4]{C}{et annexe} présente quelques fonctions C permettant la gestion du temps par une application. Nous pourrons ensuite les utiliser afin de mettre au point des attentes, ainsi que pour mesurer le temps d'exécution d'une tâche.

\section{Temps courant}

Le temps courant correspond à la perception du temps que nous avons. Il est possible, dans ce cadre, de récupérer le nombre de secondes et microsecondes écoulées depuis le début de la journée, grâce à la fonction \ccode{gettimeofday()}.


\begin{lstlisting}
#include <systime.h>
#include <unistd.h>

struct timeval {
int    tv_sec;        /* secondes */
int    tv_usec;    /* microsecondes */
};

int gettimeofday(
	struct timeval *tv,
	struct timezone *tz);
\end{lstlisting}


 Le type \ccode{struct timeval} permet de stocker cette valeur sous la forme de deux champs, le premier (\ccode{tv_sec}) représentant les secondes écoulées, et le deuxième (\ccode{tv_usec}) les microsecondes additionnelles.

Le second argument, \ccode{tz}, est obligatoire, mais n'est pas utilisé.

Exemple d'utilisation:

\begin{lstlisting}[frame=trBL]
	struct timeval t1,t2;
	struct timezone tz;
	gettimeofday(&t1,&tz);
	sleep(3);
	gettimeofday(&t2,&tz);
	printf("t1: %d, %d\n",t1.tv_sec,t1.tv_usec);
	printf("t1: %d, %d\n",t2.tv_sec,t2.tv_usec);
\end{lstlisting}

Dans cet exemple, la différence de temps au niveau des secondes, devrait être de 3, la fonction \ccode{sleep()} laissant l'application ne rien faire pendant 3 secondes.


\section{temps processeur}

Outre le temps standard, il est possible de savoir combien de temps processeur a été consommé par un processus. Cela permet de savoir exactement le temps qui a réellement été passé sur une tâche particulière, et ce malgré le fait que le processeur est partagé par les différents processus exécutés en pseudo-parallèle.

La fonction \ccode{clock()} permet de le faire, à la manière suivante:

\begin{lstlisting}[frame=trBL]
#include <time.h>

clock_t start, end;
double elapsed;

start = clock();
... /* Traitement à évaluer. */
end = clock();
elapsed = ((double) (end - start)) / CLOCKS_PER_SEC;
\end{lstlisting}

La constante \ccode{CLOCKS_PER_SEC} donne le nombre de coups d'horloge par seconde, tandis que \ccode{clock()} retourne le nombre de coups d'horloges passés depuis le lancement du programme. En combinant les deux il est aisé de trouver le nombre de secondes nécessaires à un traitement particulier.


%\section{Attente}


%int gettimeofday(struct timeval *tv, struct timezone *tz);
