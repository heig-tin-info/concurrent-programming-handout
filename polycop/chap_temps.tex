
\chapter{Temps d'ex�cution}\label{sec:temps}

\startchapter
%\section{}


\lettrine[lines=4]{C}{et annexe} pr�sente quelques fonctions C permettant la gestion du temps par une application. Nous pourrons ensuite les utiliser afin de mettre au point des attentes, ainsi que pour mesurer le temps d'ex�cution d'une t�che.

\section{Temps courant}

Le temps courant correspond � la perception du temps que nous avons. Il est possible, dans ce cadre, de r�cup�rer le nombre de secondes et microsecondes �coul�es depuis le d�but de la journ�e, gr�ce � la fonction \ccode{gettimeofday()}.


\begin{lstlisting}
#include <systime.h>
#include <unistd.h>

struct timeval {
int    tv_sec;        /* secondes */
int    tv_usec;    /* microsecondes */
};

int gettimeofday(
	struct timeval *tv,
	struct timezone *tz); 
\end{lstlisting}


 Le type \ccode{struct timeval} permet de stocker cette valeur sous la forme de deux champs, le premier (\ccode{tv_sec}) repr�sentant les secondes �coul�es, et le deuxi�me (\ccode{tv_usec}) les microsecondes additionnelles.
 
Le second argument, \ccode{tz}, est obligatoire, mais n'est pas utilis�.

Exemple d'utilisation:

\begin{lstlisting}[frame=trBL]
	struct timeval t1,t2;
	struct timezone tz;
	gettimeofday(&t1,&tz);
	sleep(3);
	gettimeofday(&t2,&tz);
	printf("t1: %d, %d\n",t1.tv_sec,t1.tv_usec);
	printf("t1: %d, %d\n",t2.tv_sec,t2.tv_usec);
\end{lstlisting}

Dans cet exemple, la diff�rence de temps au niveau des secondes, devrait �tre de 3, la fonction \ccode{sleep()} laissant l'application ne rien faire pendant 3 secondes.


\section{temps processeur}

Outre le temps standard, il est possible de savoir combien de temps processeur a �t� consomm� par un processus. Cela permet de savoir exactement le temps qui a r�ellement �t� pass� sur une t�che particuli�re, et ce malgr� le fait que le processeur est partag� par les diff�rents processus ex�cut�s en pseudo-parall�le.

La fonction \ccode{clock()} permet de le faire, � la mani�re suivante:

\begin{lstlisting}[frame=trBL]
#include <time.h>

clock_t start, end;
double elapsed;

start = clock();
... /* Traitement � �valuer. */
end = clock();
elapsed = ((double) (end - start)) / CLOCKS_PER_SEC;
\end{lstlisting}

La constante \ccode{CLOCKS_PER_SEC} donne le nombre de coups d'horloge par seconde, tandis que \ccode{clock()} retourne le nombre de coups d'horloges pass�s depuis le lancement du programme. En combinant les deux il est ais� de trouver le nombre de secondes n�cessaires � un traitement particulier.


%\section{Attente}


%int gettimeofday(struct timeval *tv, struct timezone *tz); 

